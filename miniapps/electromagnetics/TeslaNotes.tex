\documentclass[12pt]{article}
\usepackage{amsmath}

\providecommand{\A}{\mbox{\bf A}}
\providecommand{\B}{\mbox{\bf B}}
\providecommand{\E}{\mbox{\bf E}}
\providecommand{\D}{\mbox{\bf D}}
\renewcommand{\H}{\mbox{\bf H}}
\providecommand{\J}{\mbox{\bf J}}
\providecommand{\K}{\mbox{\bf K}}
\providecommand{\M}{\mbox{\bf M}}
\providecommand{\F}{\mbox{\bf F}}
\providecommand{\W}{\mbox{\bf W}}
\providecommand{\sE}{\overline{\sigma}}
\providecommand{\sM}{\overline{\sigma}_M}
\providecommand{\x}{\vec{x}}

\newcommand{\refEq}[1]{(\ref{eq:#1})}
\newcommand{\refSec}[1]{Section~\ref{sec:#1}}
\newcommand{\Div}{\nabla\cdot}
\newcommand{\Curl}{\nabla\times}
\newcommand{\Grad}{\nabla}

\title{Overview of the Tesla Miniapp in MFEM}
\author{Mark L. Stowell}

\begin{document}

\maketitle

\section{Magnetostatic Equations}

The magnetostatic equations that we start from are the following:
\begin{eqnarray}
  \Curl\H &=& \J \label{eq:ampere}\\
  \Div\B &=& 0 \label{eq:mag_gauss}\\
  \B &=& \mu\H+\mu_0\M \label{eq:const}
\end{eqnarray}
Where \refEq{ampere} is Amp\`ere's Law, \refEq{mag_gauss} is Gauss's
Law for Magnetism, and \refEq{const} is a somewhat atypical way to
write the Constitutive Relation between $\B$ and $\H$. The
constitutive relation used here follows ``Classical Electrodynamics''
3rd edition by J.D. Jackson and uses $\M$, measured in A/m, to
represent the magnetization of a permanent magnet. Some sources would
instead use $\B_r=\mu_0\M$ to represent a residual magnetization,
measured in tesla. These conventions are, of course, mathematically
equivalent but the choice made in this miniapp does seem a bit odd as
I look at it now.

These equations can be combined if we make use of the fact that
$\Div\B=0$ implies $\B=\Curl\A$ for some vector potential $\A$. This leads to:
\[\Curl\left(\mu^{-1}\Curl\A\right) = \J + \Curl\left(\mu^{-1}\mu_0\M\right)\]
This equation supports a current source density, a permanent
magnetization, surface current boundary conditions, and fixed $\A$
boundary condition which can be used to apply an external magnetic
field.

There also exists a special case in magnetostatics when the current
density is equal to zero. In this case $\Curl\H=0$ which implies that
the magnetic field can be computed as $\H=-\Grad\Phi_M$. This leads
to the scalar potential formulation which we will not consider further
except to say that the electrostatic solver, named {\tt volta}, can be
adapted to model such situations.

\section{The {\tt tesla} Miniapp}

The {\tt tesla} miniapp models the magnetostatic equation for the
magnetic vector potential $\A$. It includes source terms derived from
a volumetric current source $\J$, magnetization vector $\M$, or
surface currents $\K$.
\begin{eqnarray*}
  \Curl\left(\mu^{-1}\Curl\A\right) &=& \J +\Curl\left(\mu^{-1}\mu_0\M\right)\\
  \hat{n}\times\left(\mu^{-1}\Curl\A\right) &=& \hat{n}\times\K
\end{eqnarray*}
The magnetic vector potential will be approximated in H(Curl) so that
the left hand side operator is well defined.
\[\A\approx\sum_ia_i\W_i(\x)\]
Inserting this into the left hand side of the equation and integrating
the resulting equation against each H(Curl) basis function leads to
the following weak form:
\begin{eqnarray*}
  \int_\Omega\W_i(\x)\cdot\left[\Curl\left(\mu^{-1}\Curl\A\right)\right]\,d\Omega
  &\approx&
  \int_\Omega\W_i(\x)\cdot\left\{\Curl\left[\mu^{-1}\Curl\left(\sum_j a_j\W_j(\x)\right)\right]\right\}\,d\Omega\\
  &=&
    \sum_j a_j\left\{\int_\Omega\W_i(\x)\cdot\left[\Curl\left(\mu^{-1}\Curl\W_j(\x)\right)\right]\,d\Omega\right\}\\
\end{eqnarray*}
The expression in curly braces depends only on our material
coefficient and our basis functions. This particular integral requires
a little more manipulation to move the outermost curl operator onto
the H(Curl) basis function.
\begin{eqnarray*}
  \int_\Omega\W_i(\x)\cdot\left[\Curl\left(\mu^{-1}\Curl\W_j(\x)\right)\right]\,d\Omega
  &=&
  \int_\Omega\left(\Curl\W_i(\x)\right)\cdot\left(\mu^{-1}\Curl\W_j(\x)\right)\,d\Omega\\
  &-&  \int_\Omega\Div\left[\W_i(\x)\times\left(\mu^{-1}\Curl\W_j(\x)\right)\right]\,d\Omega \\
  &=&
  \int_\Omega\left(\Curl\W_i(\x)\right)\cdot\left(\mu^{-1}\Curl\W_j(\x)\right)\,d\Omega\\
  &-&  \int_\Gamma\hat{n}\cdot\left[\W_i(\x)\times\left(\mu^{-1}\Curl\W_j(\x)\right)\right]\,d\Gamma \\
  &=&
  \int_\Omega\left(\Curl\W_i(\x)\right)\cdot\left(\mu^{-1}\Curl\W_j(\x)\right)\,d\Omega\\
  &+&  \int_\Gamma\W_i(\x)\cdot\left[\hat{n}\times\left(\mu^{-1}\Curl\W_j(\x)\right)\right]\,d\Gamma \\
\end{eqnarray*}
The first integral remaining on the right hand side is implemented in
MFEM as a {\tt BilinearFormIntegrator} named {\tt CurlCurlIntegrator}. The second integral, the boundary integral, gives rise to a Neumann boundary condition which will be discussed further in \refSec{surf_current}. 

\subsection{Source Terms}

\subsubsection{Current Density $\J$}

The current density $\J$ requires special care. In order for the
magnetostatic equations to possess a solution $\J$ must be in the
range of the curl operator. Another way to say this is that the
divergence of $\J$ must be zero.  If $\Div\J\neq 0$ we can correct
this by adding the gradient of a scalar field. If we start with some
initial estimate of the current density which we call $\J_0$,
\begin{eqnarray*}
  \Div\left(\J_0-\Grad\Psi\right) &=& 0 \\
  \Div\Grad\Psi &=& \Div\J_0 \\
  \J & = & \J_0 - \Grad\Psi
\end{eqnarray*}
The current density $\J$ computed in this manner will be divergence
free and therefore it will be in the range of the curl operator.

Normally, in the continuous world, we simply define $\J$ directly,
however, in the discrete world we can only approximate $\J$ so we must
always perform this divergence cleaning procedure on our approximations
of $\J$. Failure to do so can lead to lack of convergence or complete
failure of the solve.

In MFEM the divergence cleaning procedure is handled by a class called
{\tt DivergenceFreeProjector} which is not a part of the MFEM library
itself. It is provided as part of a collection of convenience classes
in the {\tt miniapps/common} subdirectory.

\subsubsection{Magnetization $\M$}

The magnetization $\M$ is intended to represent permanent magnetics or
other regions of prescribed magnetization. In the Tesla miniapp $\M$
is discretized using H(Div) basis functions which allow its tangential
components to be discontinuous. Its curl appears in the magnetostatic
equations as a source term and this curl operation ensures that this
source lies in the range of the curl operator so no divergence
cleaning operation is needed for this portion of the source.

In the Tesla miniapp this source is computed and applied on lines
338-343 in the {\tt TeslaSolver::Solve()} function. The weak curl
operator is configured on lines 168-175 in the {\tt TeslaSolver}
constructor.

\subsubsection{Surface Current $\K$}
\label{sec:surf_current}

The integration by parts needed to create the weak form of the
curl-curl operators also leads to a boundary integral:
\[\int_\Gamma\W_i(\x)\cdot\left[\hat{n}\times\left(\mu^{-1}\Curl\W_j(\x)\right)\right]\,d\Gamma\]
This means that our weak curl-curl operator applied to $\A$ differs
from the continuous curl-curl operator by a surface integral of the
form:
\[\int_\Gamma\W_i(\x)\cdot\left[\hat{n}\times\left(\mu^{-1}\Curl\A\right)\right]\,d\Gamma = \int_\Gamma\W_i(\x)\cdot\left(\hat{n}\times\H\right)\,d\Gamma\]

If we do nothing to account for this boundary integral we are
implicitly setting it equal to zero which amounts to a boundary
condition on the tangential part of the magnetic field
i.e. $\hat{n}\times\H=0$. Another possibility is to set a surface
current boundary condition
i.e. $\hat{n}\times\H=\hat{n}\times\K$. This could be done by using a
{\tt ParLinerForm} object to integrate $\hat{n}\times\K$ over the
portion of the boundary where $\K$ is non-zero and adding the
resulting vector to the right hand side of the linear system.

However, this is not the approach used in the Tesla miniapp. In Tesla
we employ a trick based on the Stoke's theorem. A surface current
leads to a discontinuity in the tangential part of $\H$ on the
boundary. Similarly, a discontinuity in $\H$ leads to a discontinuity
in $\A$ on the boundary. Therefore we can set the tangential part of
$\A$ to equal $\K$ and we get the correct behavior as long as we set
the tangential part of $\A=0$ elsewhere on the boundary. To be honest
I'm not sure how valid this approach is but it does seem to work and
it can improve solver convergence. I would recommend confirming this
approach before relying on it.

\subsection{Post-Processing}

\subsubsection{Computation of $\H$}
\label{sec:h_comp}

The magnetic field $\H$ needs to have tangential continuity so we
approximate it using the H(Curl) basis:
\[\H\approx\sum_i h_i\W_i(\x)\]

Recall that the magnetic flux $\B$ is approximated using the H(Div)
basis due to the continuity of its normal component.
\[\B\approx\sum_i b_i\F_i(\x)\]

To compute $\H$ from $\B$ we make use of the constitutive equation
$\B=\mu\H$. Inserting our approximations and integrating this equation
against each H(Curl) basis function we obtain the following:
\[\sum_jh_j\int_\Omega\mu\W_i\cdot\W_j\,d\Omega =
\sum_kb_k\int_\Omega\W_i\cdot\F_k\,d\Omega\]
This set of linear equations is equivalent to the matrix equation:
\[M_1(\mu)h = M_{21}b\]

Where $M_1(\mu)$ is an H(Curl) mass matrix incorporating the material
coefficient $\mu$ which is implemented in MFEM as a {\tt
  BilinearFormIntegrator} named {\tt VectorFEMassIntegrator}. The
$M_{21}$ operator is a rectangular matrix which maps H(Div) to H(Curl)
and is also built using the {\tt VectorFEMassIntegrator} but with the 
default material coefficient which is equal to 1.

The solution of this linear system is usually obtained with a
conjugate gradient iterative solver along with a diagonal scaling
preconditioner. Since the matrix to be inverted is a mass matrix this
solution is usually very efficient involving fewer than thirty solver
iterations.

It is important to point out that an H(Curl) approximation usually has
more degrees of freedom than a comparable H(Div) approximation. In the
interior of the domain the density of degrees of freedom are
approximately equal but H(Curl) approximations tend to have more
degrees of freedom on the boundary. Consequently, this type of
conversion can produce H(Curl) approximations with poor accuracy near
the boundary. If the tangential components of $\B$ are nearly constant
within the elements adjacent to the boundary the conversion can
produce a good approximation. However, if these tangential components
vary too rapidly non-physical oscillations can occur in $\H$. To
alleviate these oscillations Dirichlet boundary conditions can be
applied during the solution of $\H$ provided that reasonable values
for $(\hat{n}\times\H)\times\hat{n}$ can be determined. In the present
magnetostatics context we can reuse any Neumann boundary conditions
used during the solution of $\A$ since these were equivalent to
setting $\hat{n}\times\H$ on the boundary.

\subsubsection{Magnetic Energy in a Region}

The {\tt tesla} miniapp does not compute the energy in the magnetic
field but such a computation should be easy to add. There are two
basic procedures for computing energy in MFEM. One involves a bilinear
form and the other a linear form. The bilinear form approach makes
sense when the energies of multiple fields will be computed with the
same operator so that the cost of building the bilinear form can be
amortized. In a magnetostatic problem the linear form approach is
likely to be more efficient.

The usual formula for magnetic energy is $u =
\frac{1}{2}\int_\Omega\H\cdot\B\,d\Omega$. There are many ways to
compute this quantity in MFEM but perhaps the most convenient is to
make use of a {\tt VectorCoefficient} and a {\tt ParLinearForm}. For
example let's assume we have a coefficient for $\mu^{-1}$ and a {\tt
  GridFunction} for $\B$ called {\tt Bgf}:
\begin{verbatim}
VectorGridFunctionCoefficient BCoef(&Bgf);
ScalarVectorProductCoefficient HCoef(muInvCoef, BCoef);
ParLinearForm Hlf(&HDivFESpace);
Hlf.AddDomainIntegrator(new VectorFEDomainIntegrator(HCoef));
Hlf.Assemble();

double energy = 0.5 * Hlf(Bgf);
\end{verbatim}
This integral can be restricted to some region, defined by a set of
element attributes, by incorporating a {\tt
  VectorRestrictedCoefficient}.

Other forms of energy such as
$\frac{1}{2}\int_\Omega\J\cdot\A\,d\Omega$ or perhaps
$\int_\Omega\M\cdot\B\,d\Omega$ could be computed in a similar manner.

\subsubsection{Torque on a Current Density}

Torque can also be defined as a volume integral so we can employ a
technique similar to the one used for the energy computation. The
important difference is that torque is a vector quantity so we will
need to integrate each of its vector components separately. This will
likely require custom coefficients but the procedure should be
straightforward. The existing vector coefficient classes {\tt
  ScalarVectorProductCoefficient} and {\tt
  VectorCrossProductCoefficient} should serve as guides for how this
can be accomplished.

\subsubsection{Torque on a Permanent Magnet}

\subsubsection{Torque on a Surface Current}

In theory a surface integral can be computed in a very similar manner
to a volume integral. However, discontinuous finite element spaces
such as H(Curl), H(Div), or L2 create a complication. Approximations
made with these discontinuous fields do not possess well defined
values on surfaces. Consequently such an integral could lack precision
or even be multi-valued.

To overcome this limitation it may be necessary to compute different
contributions to the torque in different manners and combine the
results. For example the normal component of $\B$ is well defined on
surfaces. Therefore the force $\K\times\B$ may be inaccurate but the
quantity $(\hat{n}\cdot\B)\K\times\hat{n}$ will be more reliable. To
obtain another contribution to the torque we can use the tangential
components of $\H$ as
$\mu\K\times\left[\left(\hat{n}\times\H\right)\times\hat{n}\right]$. This
of course assumes that we have an accurate representation of $\H$ on
this surface which may not be the case if the surface is an outer
boundary (see \refSec{h_comp}).

\appendix
\section{Magnetic Energy}

\begin{verbatim}
class MagneticEnergy
{
private:
   const ParGridFunction & b_;
   const ParGridFunction & h_;

public:
   MagneticEnergy(const ParGridFunction & b,
                  const ParGridFunction & h)
      : b_(b), h_(h) {}

   double ComputeEnergy()
   {
      VectorGridFunctionCoefficient h_coef(&h_);

      ParLinearForm h_lf(b_.ParFESpace());
      h_lf.AddDomainIntegrator(new VectorFEDomainLFIntegrator(h_coef));
      h_lf.Assemble();

      return 0.5 * h_lf(b_);
   }

   double ComputeEnergy(const Array<int> & elem_attr_marker)
   {
      VectorGridFunctionCoefficient h_coef(&h_);

      ParLinearForm h_lf(b_.ParFESpace());
      h_lf.AddDomainIntegrator(new VectorFEDomainLFIntegrator(h_coef),
                               const_cast<Array<int>&>(elem_attr_marker));
      h_lf.Assemble();

      return 0.5 * h_lf(b_);
   }
};
\end{verbatim}

\section{Torque}

\begin{verbatim}
class Torque
{
private:
   const ParGridFunction & b_;
   const ParGridFunction & h_;
   const ParGridFunction & j_;

public:
   Torque(const ParGridFunction & b,
          const ParGridFunction & h,
          const ParGridFunction & j)
      : b_(b), h_(h), j_(j) {}

   void ComputeTorqueOnSurface(const Array<int> &bdr_attr_marker,
                               const Vector &cent, Vector &T);
   void ComputeTorqueOnVolume(const Array<int> &vol_attr_marker,
                              const Vector &cent, Vector &T);
};
\end{verbatim}

\begin{verbatim}
void Torque::ComputeTorqueOnSurface(const Array<int> &bdr_attr_marker,
                                    const Vector &cent, Vector &trq)
{
   trq = 0.0;

   ParFiniteElementSpace * fes = b_.ParFESpace();
   ParMesh *mesh = b_.ParFESpace()->GetParMesh();
   ElementTransformation *eltrans = NULL;

   Vector b, h, ht(3), nor(3), x(3), f(3), loc_trq(3);
   loc_trq = 0.0;

   for (int i=0; i<fes->GetNBE(); i++)
   {
      const int bdr_attr = mesh->GetBdrAttribute(i);
      if (bdr_attr_marker[bdr_attr-1] == 0) { continue; }

      eltrans = fes->GetBdrElementTransformation(i);
      const FiniteElement &el = *fes->GetBE(i);

      const IntegrationRule *ir = NULL;
      if (ir == NULL)
      {
         const int order = 2*el.GetOrder() + eltrans->OrderW(); // <-----
         ir = &IntRules.Get(eltrans->GetGeometryType(), order);
      }

      for (int pi = 0; pi < ir->GetNPoints(); ++pi)
      {
         const IntegrationPoint &ip = ir->IntPoint(pi);

         eltrans->SetIntPoint(&ip);

         CalcOrtho(eltrans->Jacobian(), nor);

         double a = nor.Norml2();

         eltrans->Transform(ip, x);

         b_.GetVectorValue(*eltrans, ip, b);
         h_.GetVectorValue(*eltrans, ip, h);

         double bn = b * nor / a;
         double hn = h * nor / a;
         add(h, -hn / a, nor, ht);

         f.Set(ip.weight * bn * bn / mu0_, nor);
         f.Add(ip.weight * a * bn, ht);
         f.Add(-0.5 * ip.weight * (mu0_ * (ht * ht) + bn * bn / mu0_), nor);

         loc_trq[0] += (x[1]-cent[1]) * f[2] - (x[2]-cent[2]) * f[1];
         loc_trq[1] += (x[2]-cent[2]) * f[0] - (x[0]-cent[0]) * f[2];
         loc_trq[2] += (x[0]-cent[0]) * f[1] - (x[1]-cent[1]) * f[0];
      }
   }

   MPI_Allreduce(loc_trq, trq, 3, MPI_DOUBLE, MPI_SUM, fes->GetComm());
}
\end{verbatim}

\begin{verbatim}
void Torque::ComputeTorqueOnVolume(const Array<int> &attr_marker,
                                   const Vector &cent, Vector &trq)
{
   trq = 0.0;

   ParFiniteElementSpace * fes = b_.ParFESpace();
   ParMesh *mesh = b_.ParFESpace()->GetParMesh();
   ElementTransformation *eltrans = NULL;

   Vector b, j, x(3), f(3), t(3), loc_trq(3);
   loc_trq = 0.0;

   for (int i=0; i<fes->GetNE(); i++)
   {
      const int attr = mesh->GetAttribute(i);
      if (attr_marker[attr-1] == 0) { continue; }

      eltrans = fes->GetElementTransformation(i);
      const FiniteElement &el = *fes->GetFE(i);

      const IntegrationRule *ir = NULL;
      if (ir == NULL)
      {
         const int order = 2*el.GetOrder() + eltrans->OrderW(); // <-----
         ir = &IntRules.Get(eltrans->GetGeometryType(), order);
      }

      for (int pi = 0; pi < ir->GetNPoints(); ++pi)
      {
         const IntegrationPoint &ip = ir->IntPoint(pi);

         eltrans->SetIntPoint(&ip);

         eltrans->Transform(ip, x);

         b_.GetVectorValue(*eltrans, ip, b);
         j_.GetVectorValue(*eltrans, ip, j);

         f[0] = j[1] * b[2] - j[2] * b[1];
         f[1] = j[2] * b[0] - j[0] * b[2];
         f[2] = j[0] * b[1] - j[1] * b[0];

         t[0] = (x[1]-cent[1]) * f[2] - (x[2]-cent[2]) * f[1];
         t[1] = (x[2]-cent[2]) * f[0] - (x[0]-cent[0]) * f[2];
         t[2] = (x[0]-cent[0]) * f[1] - (x[1]-cent[1]) * f[0];

         loc_trq.Add(ip.weight * eltrans->Weight(), t);
      }
   }

   MPI_Allreduce(loc_trq, trq, 3, MPI_DOUBLE, MPI_SUM, fes->GetComm());
}
\end{verbatim}

\end{document}
